% Vypisy apod

\usepackage{amsmath}
\newenvironment{eqad}{\begin{equation}\begin{aligned}}{\end{aligned}\end{equation}}
\newenvironment{eqad*}{\begin{equation*}\begin{aligned}}{\end{aligned}\end{equation*}}

\newcommand{\vl}[2]{$#1\,\text{#2}$}
\newcommand{\vle}[2]{#1\,\text{#2}}
\newcommand{\ela}[2]{$#1 = #2$}
\newcommand{\el}[3]{$#1 = #2\,\text{#3}$}
\newcommand{\ii}{\noindent}
\newcommand{\iib}[1]{\bigskip \noindent \textbf{#1}}

% Use \cref and \Cref instead.
%\newcommand{\refeq}[1]{Equation~\ref{eq:#1}}
%\newcommand{\reffig}[1]{Figure~\ref{fig:#1}}
%\newcommand{\refsec}[1]{Section~\ref{sec:#1}}
%\newcommand{\refssec}[1]{Subsection~\ref{ssec:#1}}
%\newcommand{\refalg}[1]{Algorithm~\ref{alg:#1}}
%
%\newcommand{\refcite}[1]{\textcolor{lightgray}{[#1]}}
%
%\newcommand{\labeleq}[1]{\label{eq:#1}}
%\newcommand{\labelfig}[1]{\label{fig:#1}}
%\newcommand{\labelsec}[1]{\label{sec:#1}}
%\newcommand{\labelssec}[1]{\label{ssec:#1}}
%\newcommand{\labelalg}[1]{\label{alg:#1}}


% TODO

\usepackage{tocloft}
\usepackage{xcolor}
\usepackage{hyperref}

% Toggle TODOs
\newif\ifshowTODO
\showTODOfalse
\newif\iflinkTODO
\linkTODOfalse

\newcommand{\listTODOname}{TODO list}
\newlistof[subsection]{TODO}{todos}{\listTODOname}

%\newcommand{\TODO}[1]{}
%\newcommand{\TODOd}[1]{}

\newcommand{\TODO}[1]{%
\ifshowTODO
\refstepcounter{TODO}
\iflinkTODO
\textcolor{red}{\hyperref[todolist]{\textbf{TODO: #1}}}
\else
\textcolor{red}{\textbf{TODO: #1}}
\fi
\addcontentsline{todos}{TODO}{\thesubsection\quad\textcolor{red}{#1}}
\fi}

\newcommand{\TODOd}[1]{%
\ifshowTODO
\refstepcounter{TODO}
\textcolor{green!70!blue}{\textbf{TODO: #1}}
\addcontentsline{todos}{TODO}{\thesubsection\quad\textcolor{green!70!blue}{#1}}
\fi}

%\addcontentsline{todos}{TODO}{\protect\numberline{\theTODO}\qquad\textcolor{green!70!blue}{#1}}}

%https://tex.stackexchange.com/questions/118283/transforming-numbers-to-alph-numbers
\newcommand{\makeAlph}[1]{%
  \ifcase #1?\or a\or b\or c\or d\or e\or f\or g\or h\or
    i\or j\or k\or l\or m\or n\or o\or p\or q\or r\or
    s\or t\or u\or v\or w\or x\or y\or z\else ?\fi
}

\newcommand{\eA}[1]{\makeAlph{#1})}
\newcommand{\eL}{\the\value{enumi}}

% Kodovani (české znaky)
\usepackage[utf8]{inputenc} %for linux users

% TikZ
\usepackage{tikz}
\usetikzlibrary{math}
\usetikzlibrary{patterns} % výplně
\usetikzlibrary{intersections} % name path
\usepackage{tikzscale} % \includegraphics[]{*.tikz}
\usepackage{ifthen} % compare in tikz

%Zisset
\newcommand{\Zisset}[2]{
    \ifthenelse{#1 > 0}{
        #2
    }{}
}

%Zissetelse
\newcommand{\Zissetelse}[3]{
    \ifthenelse{#1 > 0}{
        #2
    }{
        #3
    }
}

% axis environment (TikZ)
\usepackage{pgfplots}

% MATLAB colors
\definecolor{matlab1}{rgb}{0, 0.447, 0.741}         % blue
\definecolor{matlab2}{rgb}{0.85, 0.325, 0.098}      % orange
\definecolor{matlab3}{rgb}{0.929, 0.694, 0.125}     % yellow
\definecolor{matlab4}{rgb}{0.494, 0.184, 0.556}     % purple
\definecolor{matlab5}{rgb}{0.466, 0.674, 0.188}     % green
\definecolor{matlab6}{rgb}{0.301, 0.745, 0.933}     % cyan
\definecolor{matlab7}{rgb}{0.635, 0.078, 0.184}     % maroon

% Pseudocode
\usepackage{algorithm}
\usepackage{algorithmic}

\floatname{algorithm}{Algorithm}
\renewcommand{\algorithmicrequire}{\textbf{Input:}}
\renewcommand{\algorithmicensure}{\textbf{Output:}}

% Notes
\newif\ifshowNOTE
\showNOTEfalse
\newcommand{\note}[1]{\ifshowNOTE\textcolor{lightgray}{[#1]}\fi}
%\newcommand{\blocknote}[1]{\ifshowNOTE\ii{\color{lightgray}\fbox{\parbox{\linewidth}{#1}}}\fi}
\usepackage{tcolorbox}
% This is fbox with rounded corners and frame width same as size=normal
\newcommand{\blocknote}[1]{\ifshowNOTE\ii\begin{tcolorbox}[colback=white,colframe=lightgray,size=fbox,arc=1mm,boxrule=0.5mm]{\color{gray}\parbox{\linewidth}{#1}}\end{tcolorbox}\fi}

% Twofigure
\usepackage{subcaption}
\newlength{\twosubht}
\newsavebox{\twosubbox}

\newcommand{\includetwofigure}[9]{
% img1, caption1, label1, img2, caption2, label2, caption, label
% Za posledním argumentem musí být prázdná řádka, jinak figure křičí!

% preliminary
\sbox\twosubbox{%
  \resizebox{\dimexpr.9\linewidth-1em}{!}{%
    \includegraphics[width=0.5\linewidth]{#1}%
    \includegraphics[width=0.5\linewidth]{#4}%
  }%
}
\setlength{\twosubht}{\ht\twosubbox}

% typeset

\centering

\subcaptionbox{#2#3}{%
  \includegraphics[height=\twosubht]{#1}%
}\quad
\subcaptionbox{#5#6}{%
  \includegraphics[height=\twosubht]{#4}%
}

\caption{#7}
#8
}

% Tableau colors (Matplotlib)
\definecolor{tab:blue}{rgb}{0.122, 0.467, 0.706}    % blue
\definecolor{tab:orange}{rgb}{1.000, 0.498, 0.055}  % orange
\definecolor{tab:green}{rgb}{0.173, 0.627, 0.173}   % green
\definecolor{tab:red}{rgb}{0.839, 0.153, 0.157}     % red
\definecolor{tab:purple}{rgb}{0.580, 0.404, 0.741}  % purple
\definecolor{tab:brown}{rgb}{0.549, 0.337, 0.294}   % brown
\definecolor{tab:pink}{rgb}{0.890, 0.467, 0.761}    % pink
\definecolor{tab:gray}{rgb}{0.498, 0.498, 0.498}    % gray
\definecolor{tab:olive}{rgb}{0.737, 0.741, 0.133}   % olive
\definecolor{tab:cyan}{rgb}{0.090, 0.745, 0.812}    % cyan

%% ^ Position redefinition
\definecolor{matplotlib1}{rgb}{0.122, 0.467, 0.706}     % blue
\definecolor{matplotlib2}{rgb}{1.000, 0.498, 0.055}     % orange
\definecolor{matplotlib3}{rgb}{0.173, 0.627, 0.173}     % green
\definecolor{matplotlib4}{rgb}{0.839, 0.153, 0.157}     % red
\definecolor{matplotlib5}{rgb}{0.580, 0.404, 0.741}     % purple
\definecolor{matplotlib6}{rgb}{0.549, 0.337, 0.294}     % brown
\definecolor{matplotlib7}{rgb}{0.890, 0.467, 0.761}     % pink
\definecolor{matplotlib8}{rgb}{0.498, 0.498, 0.498}     % gray
\definecolor{matplotlib9}{rgb}{0.737, 0.741, 0.133}     % olive
\definecolor{matplotlib10}{rgb}{0.090, 0.745, 0.812}    % cyan

% Inline list
\usepackage[inline]{enumitem}
%\newlist{inlinelist}{enumerate*}{1}%
%\setlist[inlinelist,1]{label=(\roman*), ref=\roman*}

% https://tex.stackexchange.com/questions/296098/counting-items-in-itemize-or-enumerate-environments
% New inline list that we can count.
% One optional argument: name of label to obtain number of elements
\makeatletter
\newenvironment{inlinelist}[1][]
  {\def\countname{#1}\begin{enumerate*}[label=(\roman*), after=]
     %\setcounter{countlist}{0}%
     %\let\olditem\item%
     %\renewcommand{\item}{\stepcounter{countlist}\olditem\thecountlist}% This does not increment in inline list.
     %\item \thecountlist
  }
  {%
    \renewcommand{\@currentlabel}{\arabic{enumi}}%
    \label{\countname}%
    \end{enumerate*}%
}% Comments are here to kinda fix the spacing after the inlinelist.
\makeatother

\newcommand{\numitems}[1]{\numToWord{\getrefnumber{#1}}}

% Counting the items
% https://tex.stackexchange.com/questions/296098/counting-items-in-itemize-or-enumerate-environments
%\usepackage{xpatch}
%\usepackage{refcount}

%\newcommand{\numitems}[1]{\getrefnumber{#1}}
%\newcounter{itemcntr}

%\AtBeginEnvironment{itemize}{%
%\setcounter{itemcntr}{0}%
%\xapptocmd{\item}{\refstepcounter{itemcntr}}{}{}%
%}

% https://tex.stackexchange.com/questions/227858/howto-convert-roman-into-arabic-numeral
\usepackage{xparse}
\ExplSyntaxOn
\DeclareExpandableDocumentCommand{\arabicnumeral}{m}
 {
  \int_from_roman:n { #1 }
 }
\ExplSyntaxOff

% Numbers to string
% Similar to makeAlph above
% TODO: Tohle prostě nefunguje. Převod jde, ale \ref vrací něco, co se nedá převést. I když je to pořád ii.
\newcommand{\numToWord}[1]{%
  \ifnum\arabicnumeral{#1}=-1
    \ifcase #1?\or one\or two\or three\or four\or five\or six\or seven\or eight\or
    nine\or ten\else #1\fi
  \else
    \ifcase \arabicnumeral{#1}?\or one\or two\or three\or four\or five\or six\or seven\or eight\or
    nine\or ten\else \arabicnumeral{#1}\fi
  \fi
}

% Clever references
% Must be loaded after hyperref!
\usepackage[capitalise]{cleveref}
\crefformat{algorithm}{Alg.~#2#1#3}
\Crefformat{algorithm}{Algorithm~#2#1#3}
\crefformat{section}{Sec.~#2#1#3}
\Crefformat{section}{Section~#2#1#3}
\crefformat{subsection}{Sec.~#2#1#3}
\Crefformat{subsection}{Section~#2#1#3}
\crefformat{subsubsection}{Sec.~#2#1#3}
\Crefformat{subsubsection}{Section~#2#1#3}
